





\begin{frame}
  \frametitle{熱流とエントロピー生成率とのトレードオフ}
  \begin{theorem}[熱流とエントロピー生成率とのトレードオフ]
    以上のsettingの下で, 
    \begin{equation}
      \sum_{\nu=1}^{k}|J_{\nu}^q|\leq\sqrt{\Theta^{(1)}\dot{\sigma}}. 
    \end{equation}
    ここで, 
    \begin{equation}
      \Theta^{(1)}\coloneqq \frac{1}{c_0}\sum_{\mu}\sum_{i\neq j}\left( \Delta E_{i}^{\mu} \right)^2\left( R_{ij}^\mu p_j+R_{ji}^{\mu}p_i \right)\eqqcolon \sum_{\mu}\Theta_{\mu}^{(1)}, 
    \end{equation}
    であり, 
    \begin{equation}
      c_0=\frac{8}{9},\quad \Delta E_{i}^{\mu}\coloneqq E_{m}^i-\braket{E}_{t,w^{-m}}=E_{m}^i-\frac{\int\dd{w'^{m}}E(w'^{m},w^{-m})p_{w'^{m},w^{-m},t}}{\int\dd{w'^m}p_{w'^{m},w^{-m},t}}, 
    \end{equation}
    とする. %\footnote{$\braket{E}_{t,w^{-m}}$は$m$番目の粒子以外の状態をfixして, そのもとで系のエネルギー期待値を計算する操作を表している. }. 
  \end{theorem}
\end{frame}