



\section{\texorpdfstring{$\Theta^{(i)}$の評価}{Theta^(i)の評価}}

\subsection{\texorpdfstring{$\Theta^{(1)}$の評価}{Theta^(1)の評価}}

\begin{frame}[shrink=15]
  \frametitle{\texorpdfstring{$\Theta^{(1)}$の評価}{Theta^(1)の評価}}
  \begin{align}
  \Theta_{\mu}^{(1)}&=\frac{1}{c_0}\sum_{w\neq w'}\left( \Delta E_{w}^{\mu} \right)^2\left( R_{ww'}^\mu p_{w'}+R_{w'w}^{\mu}p_w \right)\\
  &=\frac{1}{c_0}\sum_{w}\left( \sum_{w'\neq w}\left( \Delta E_{w}^{\mu} \right)^2 R_{ww'}^{\mu}p_{w'}+\left( \Delta E_{w}^{\mu} \right)^2(-R_{ww}^\mu)p_w \right)\\
  &=\frac{1}{c_0}\left( \sum_{w', w}\left( \Delta E_{w}^{\mu} \right)^2 R_{ww'}^{\mu}p_{w'}-2\sum_{w}\left( \Delta E_{w}^{\mu} \right)^2R_{ww}^\mu p_w \right)\\
  &=\frac{1}{c_0}\left( \sum_{w}\left( \Delta E_{w}^{\mu} \right)^2 \left[ \frac{\dd{}}{\dd{t}} \right]_{\mu}p_w
  +2\sum_{w}\left( \Delta E_{w}^{\mu} \right)^2|R_{ww}^\mu |p_w \right)\\
  &\leq \frac{1}{c_0}\sum_{w^{-m}}p_{w^{-m}}\left( \left[ \frac{\dd{}}{\dd{t}} \right]_\mu\braket{(E_{w}^{\mu})^2}_{t,w^{-m}} + 2R_{max}\braket{(E_{w}^\mu)^2}_{t,w^{-m}}\right). 
\end{align}
上式で1項目が発散することはなさそうなので, $R_{max}$が有界であれば$\Theta^{(1)}$も有界だとわかる.
  % \Gg{\left[ \frac{\dd{}}{\dd{t}} \right]_{\mu}p_w\coloneqq \sum_{w'}R_{ww'}p_{w'}}
  % \Gg{2項目, R_{max}\geq |R_{ww}^\mu |}
\end{frame}

\subsection{\texorpdfstring{$\Theta^{(2)}$と熱伝導度}{Theta^(2)と熱伝導度}}
\begin{frame}[shrink=15]
  \frametitle{\texorpdfstring{$\Theta^{(2)}$と熱伝導度}{Theta^(2)と熱伝導度}}
    線形応答領域において$\Theta^{(2)}$が熱伝導度に対応する. 
    \begin{figure}[H]
    \centering
    \includegraphics[keepaspectratio, scale=0.04]{figures/tradeoff_power-efficiency_theta2.drawio.png}
    \caption{状況設定}\label{fig:power-efficiency_theta2}
  \end{figure}
    $\Delta\beta >0$として, 逆温度$\beta-\Delta\beta$の熱浴と平衡状態にあった系について, 熱浴を逆温度$\beta$のものに取り替えると, 熱浴に$J^q$だけの熱流が流れる.
    このとき, 現象論的にはFourierの法則より
    \begin{equation}
      J^q=\kappa \Delta\beta, 
    \end{equation} 
    と表されると期待される. 
    ここで$\kappa$は熱伝導度である. 
\end{frame}

\begin{frame}[shrink=15]
  \frametitle{\texorpdfstring{$\Theta^{(2)}$と熱伝導度}{Theta^(2)と熱伝導度}}
  このとき, $p_i^\beta$を温度$\beta$のcanonical分布として$J^q$を式変形すると, 
  \begin{align}
    J^q&=\sum_{ij}(E_i-E_j)R_{ji}p_{i}^{\beta-\Delta\beta}\\
    \Gg{\Delta\beta で展開}&=\sum_{ij}(E_i-E_j)R_{ji}p_i^{\beta}\left( 1+(E_i-\braket{E}^\beta)\Delta\beta \right)+{O}((\Delta\beta)^2)\\
    \Gg{詳細つりあい}&\simeq \sum_{ij}(E_i-E_j)R_{ji}p_i^{\beta}(E_i-\braket{E}^\beta)\Delta\beta \\
    &=\frac{1}{2}\sum_{ij}(E_i-E_j)R_{ji}p_i^{\beta}(E_i-\braket{E}^\beta)\Delta\beta+\frac{1}{2}\sum_{ij}(E_j-E_i)R_{ij}p_j^{\beta}(E_j-\braket{E}^\beta)\Delta\beta\\
    &=\frac{1}{2}\sum_{ij}(E_i-E_j)\left[ R_{ji}p_i^{\beta}(E_i-\braket{E}^\beta)-R_{ij}p_j^{\beta}(E_j-\braket{E}^\beta) \right]\Delta\beta\\
    \Gg{詳細つりあい}&=\frac{1}{2}\sum_{ij}(E_i-E_j)^2R_{ji}p_i^{\beta}\Delta\beta\\
    &=\Theta^{(2)}\Delta\beta, 
  \end{align}
  と変形されるので, 
  \begin{equation}
    \kappa=\Theta^{(2)}
  \end{equation}
  となり, $\Theta^{(2)}$に物理的意味が与えられる. 
\end{frame}