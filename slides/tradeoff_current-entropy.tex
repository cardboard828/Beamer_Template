
\section{熱流とエントロピー生成率のトレードオフ}



\begin{frame}
  {目次}
   \tableofcontents[currentsection]
\end{frame}

\begin{frame}
  \frametitle{状況設定}
  \begin{figure}[H]
    \centering
    \includegraphics[keepaspectratio, scale=0.03]{figures/tradeoff_current-entropy.drawio.png}
    \caption{系の概念図}\label{fig:tradeoff_current-entropy}
  \end{figure}
  
\end{frame}

\begin{frame}
  \frametitle{状況設定}
  \begin{figure}
    \begin{subfigure}{.4\textwidth}
        \centering
        \includegraphics[width = \linewidth]{figures/tradeoff_current-entropy.drawio.png}
        \caption{系全体}
    \end{subfigure}
    \begin{subfigure}{.4\textwidth}
        \centering
        \includegraphics[width = \linewidth]{figures/tradeoff_current-entropy_separated.drawio.png}
        \caption{重ね合わせの要素}
    \end{subfigure}
    \caption{概念図}
\end{figure}
\end{frame}




\begin{frame}
  \frametitle{熱流とエントロピー生成率とのトレードオフ}
  \begin{theorem}[熱流とエントロピー生成率とのトレードオフ]\label{thm:tradeoff_current-entropy}
    以上のsettingの下で, 
    \begin{equation}
      \sum_{\nu=1}^{k}|J_{\nu}^q|\leq\sqrt{\Theta^{(1)}\dot{\sigma}}. 
    \end{equation}
    ここで, 
    \begin{equation}
      \Theta^{(1)}\coloneqq \frac{1}{c_0}\sum_{\mu}\sum_{i\neq j}\left( \Delta E_{i}^{\mu} \right)^2\left( R_{ij}^\mu p_j+R_{ji}^{\mu}p_i \right)\eqqcolon \sum_{\mu}\Theta_{\mu}^{(1)}, 
    \end{equation}
    であり, 
    \begin{equation}
      c_0=\frac{8}{9},\quad \Delta E_{i}^{\mu}\coloneqq E_{m}^i-\braket{E}_{t,w^{-m}}%=E_{m}^i-\frac{\int\dd{w'^{m}}E(w'^{m},w^{-m})p_{w'^{m},w^{-m},t}}{\int\dd{w'^m}p_{w'^{m},w^{-m},t}}, 
    \end{equation}
    とする. %\footnote{$\braket{E}_{t,w^{-m}}$は$m$番目の粒子以外の状態をfixして, そのもとで系のエネルギー期待値を計算する操作を表している. }. 
  \end{theorem}
\end{frame}


\begin{frame}
  \frametitle{局所詳細つりあいがある場合}
  \begin{theorem}[熱流とエントロピー生成率とのトレードオフ(局所詳細つりあいがある場合)]\label{thm:tradeoff_current-entropy_detailed-balance}
    さらに, 任意の$w, w', \nu$について局所詳細つりあい
    \begin{equation}
      R_{ij}^{\mu}e^{-\beta_{\nu}E_j}=R_{ji}^{\mu}e^{-\beta_{\nu}E_i}, 
    \end{equation}
    が成立するとき, 
    \begin{equation}
      \sum_{\nu=1}^{k}|J_{\nu}^q|\leq\sqrt{\Theta^{(2)}\dot{\sigma}}. 
    \end{equation}
    ここで, 
    \begin{equation}
      \Theta^{(2)}\coloneqq \frac{1}{2}\sum_{i\neq j}\left( E_i-E_j \right)^2 R_{ij}p_j\eqqcolon \sum_{\mu}\Theta_{\mu}^{(2)}, 
    \end{equation}
    最後の$\eqqcolon$では$R_{ij}$の分解を使っている. 
  \end{theorem}
\end{frame}

